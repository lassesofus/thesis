\section{Extended number of planning steps}
\label{app:long_zero_shot_planning}
To examine whether additional planning steps alleviate performance limitations, the number of planning steps is increased from 5 to 10. As shown in Figure~\ref{fig:exp01_zero_shot_long}, doubling the number of steps yields mixed results across axes: $x$-axis performance improved substantially (final error from 6.8\,cm to 1.3\,cm, corresponding to reductions of 13.2\,cm and 18.7\,cm respectively), $y$-axis performance remained essentially unchanged (final error from 2.2\,cm to 2.4\,cm, reductions of 17.8\,cm and 17.6\,cm), while $z$-axis performance degraded (final error from 3.2\,cm to 5.2\,cm, reductions of 16.8\,cm and 14.8\,cm). The improvement along the $x$-axis suggests that when latent-space gradients align well with task geometry, additional planning iterations can provide meaningful benefits. However, the degradation along the $z$-axis is consistent with the structural nature of failure modes identified above: when the energy landscape admits multiple local minima, extended optimization may allow the planner to settle into different basins across episodes, potentially increasing rather than reducing trajectory variance. The unchanged $y$-axis performance indicates that for well-aligned axes, five steps already achieve near-optimal convergence.


\begin{figure}
    \centering
    \includegraphics[width=\linewidth]{Pictures/Experiments/01_zero_shot_single_axis_reaching/consolidated_analysis_long_planning.png}
    \caption{Extended planning (10 steps) for zero-shot reaching. Each column shows reaching along a different axis ($x$, $y$, $z$): start and goal frames (row 1), Cartesian error (row 2), and latent distance (row 3). Error bars: $\pm1$ std ($N=10$).}
    \label{fig:exp01_zero_shot_long}
\end{figure}
