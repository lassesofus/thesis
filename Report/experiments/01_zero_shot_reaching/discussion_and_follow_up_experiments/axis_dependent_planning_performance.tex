\subsubsection{Axis-dependent planning performance}
The marked disparity in planning performance across Cartesian axes seen in Figure~\ref{fig:exp01_zero_shot} is possibly driven by differences in visual observability induced by the fixed monocular camera setup. Owing to the frontal-lateral viewpoint, motion along different axes projects unevenly into image space. Displacement along the $y$-axis (toward or away from the camera) produces strong depth cues in the form of apparent scale changes, whereas $x$-axis motion results primarily in subtle lateral shifts with minimal change in object scale. Motion along the $z$-axis induces vertical displacement that can alter the arm’s visibility within the frame, but lacks the smooth scale variation associated with depth changes. 

Further insight is provided by the executed trajectories shown in Figure~\ref{fig:exp01_zero_shot}. Planning along the $x$-axis frequently leads to trajectories with substantial unintended components along both $y$- and $z$-axes. One plausible explanation is a bias in the learned action prior inherited from the action-conditioned post-training of the predictor on the DROID dataset. If the training distribution underrepresents pure lateral motions, the model may favor actions aligned with more frequently observed directions. A more thorough analysis of the action distribution in the DROID dataset could provide further support for this claim. Finally, the high variability in $z$-axis performance, with final errors spanning 8.5\,cm to 26.7\,cm, suggests potential ambiguity in how elevation changes are encoded in monocular visual representations.