\subsubsection{Extended number of planning steps}
To examine whether additional planning steps alleviate performance limitations, the number of planning steps is increased from 5 to 10. As shown in Figure~\ref{fig:exp01_zero_shot_long}, doubling the number of steps does not yield substantial additional reductions in Cartesian error. For the $x$-axis, the error increases slightly from 16.2\,cm at step 5 to 17.1\,cm at step 10. For the $y$-axis, performance similarly degrades from 9.2\,cm to 10.0\,cm. The $z$-axis shows improvement on average from 15.3\,cm to 13.3\,cm, with reduced variance (from $\pm5.1$\,cm to $\pm2.9$\,cm). However, still none of the 30 episodes reach the 5\,cm success threshold. These results indicate that performance plateaus around step 5, with additional planning steps providing no substantial benefit and in some cases causing slight regression. This suggests that the limiting factor in the current setting is not the number of planning steps. 

\begin{figure}[H]
    \centering
    \includegraphics[width=\linewidth]{Pictures/Experiments/01_zero_shot_single_axis_reaching/consolidated_analysis_long_planning.png}
    \caption{Extended planning (10 steps) for zero-shot reaching. Each column shows reaching along a different axis ($x$, $y$, $z$): start and goal frames (row 1), Cartesian error (row 2), and latent distance (row 3). Error bars: $\pm1$ std ($N=10$).}
    \label{fig:exp01_zero_shot_long}
\end{figure}


% See plot in /home/s185927/thesis/experiments/01_zero_shot_single_axis_reaching/left_cam_long_planning/consolidated_analysis_long_planning.png