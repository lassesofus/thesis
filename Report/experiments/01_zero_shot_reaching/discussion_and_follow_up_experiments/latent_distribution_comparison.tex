\subsubsection{Latent distribution comparison: DROID vs.\ simulation}

An additional observation is that planning performance in the simulated RoboHive environment appears substantially weaker than the similar planning examples reported in the original V-JEPA~2 work. In particular, even along axes where monotonic progress is observed, the magnitude of improvement remains limited, and no episodes reach the task success threshold under zero-shot deployment.

One plausible explanation is that the simulated visual observations constitute a distribution shift relative to the DROID dataset used during action-conditioned post-training of the V-JEPA predictor. Although the encoder preserves coarse spatial structure, as evidenced by the strong latent--physical distance correlations reported in Figure \ref{fig:latent_physical_alignment}, the resulting representations may differ in ways that are critical for action-conditioned prediction and planning. If simulated observations are embedded into regions of representation space whose geometric structure differs from those induced by real-world DROID data, the action-conditioned predictor may operate in an effectively out-of-distribution regime.

To assess the sim-to-real domain gap in the pretrained representation space, 1000 RGB frames from DROID episodes (left exocentric camera) and 1000 frames from RoboHive simulation rollouts were randomly sampled. All frames were encoded using the frozen V-JEPA 2 encoder, and spatial features were mean-pooled to obtain frame-level embeddings of dimension $D = 1408$. The resulting embeddings were analyzed using principal component analysis (PCA) and pairwise distance statistics.

Figure~\ref{fig:latent_distribution_pca} shows the first two principal components of the combined embedding space. DROID and simulation embeddings occupy distinct regions that are clearly separated along the first principal component, which alone explains 42.3\% of the total variance. This indicates that domain identity constitutes a dominant axis of variation in the pretrained V-JEPA-2 representation space. The first two principal components together account for 47.4\% of the total variance, suggesting that the observed separation reflects a systematic representational offset rather than a projection artifact.

To quantify the degree of separation between domains, mean pairwise $\ell_2$ distances are computed within and across domains. Let $\bar{d}_{\text{intra}}^{\text{DROID}}$ and $\bar{d}_{\text{intra}}^{\text{sim}}$ denote the mean intra-domain distances for DROID and simulation embeddings respectively, and let $\bar{d}_{\text{inter}}$ denote the mean distance between randomly paired DROID and simulation embeddings. The domain gap ratio is defined as
\begin{equation}
    \text{Gap ratio} = \frac{\bar{d}_{\text{inter}}}{\frac{1}{2}\left(\bar{d}_{\text{intra}}^{\text{DROID}} + \bar{d}_{\text{intra}}^{\text{sim}}\right)}.
\end{equation}
A ratio close to 1.0 indicates substantial overlap between domains, while larger values indicate increasing separation. In this analysis, the gap ratio is 1.71, indicating that inter-domain distances are substantially larger than intra-domain distances. This is consistent with a systematic offset between simulation and real-world representations rather than complete disjointness.


Taken together, these results suggest that while the pretrained encoder extracts some domain-invariant features that generalize across sim-to-real settings, simulated observations are embedded into latent regions whose geometry differs meaningfully from those encountered during action-conditioned training on DROID data. As a consequence, the action-conditioned predictor must extrapolate to latent geometries for which its learned dynamics may be misaligned. This representational mismatch provides a plausible explanation for the degraded zero-shot planning performance observed in Figure \ref{fig:exp01_zero_shot}, even in cases where planning behavior remains stable and latent distances decrease monotonically.


\begin{figure}[H]
    \centering
    \includegraphics[width=\linewidth]{Pictures/Experiments/01_zero_shot_single_axis_reaching/latent_distribution_comparison/pca_analysis.png}
    \caption{PCA projection of V-JEPA~2 latent embeddings for randomly sampled frames from DROID (blue) and simulation (orange). (a) Scatter plot of the first two principal components, showing a clear separation between domains along the dominant principal axis, with simulation embeddings occupying a compact region offset from the DROID distribution. (b) Variance explained by the first 10 principal components, with cumulative variance shown in red.}

    \label{fig:latent_distribution_pca}
\end{figure}


% See plots in /home/s185927/thesis/experiments/01_zero_shot_single_axis_reaching/follow_up_experiments/latent_distribution_comparison
