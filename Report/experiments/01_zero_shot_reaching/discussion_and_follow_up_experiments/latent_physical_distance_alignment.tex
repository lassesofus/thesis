The relationship between latent-space error and Cartesian position error during planning reveals a non-trivial mismatch. Reaching along the $y$-axis yields the strongest reduction in Cartesian error despite exhibiting a small reduction in latent-space distance over the five planning steps (from 0.52 to 0.51), including a transient increase at the first step. In contrast, planning along the $x$- and $z$-axes yields monotonic (on average) decreases in latent distance (0.57 to 0.45 and 0.48 to 0.38, respectively), yet results in weaker Cartesian progress.

This relationship between the physical Euclidean distance $\lVert p_k - p_g \rVert_2$ and the mean latent $\ell_1$ distance $\frac{1}{TD}\lVert z_k - z_g \rVert_1$ was further investigated. Specifically, single-axis trajectories were executed over 4.5 seconds using IK control with a 20 cm offset along each axis and sampled at 30 FPS. Five identical episodes were recorded per axis, with each episode using its own final frame as the goal for computing latent distances. Figure~\ref{fig:latent_physical_alignment} shows scatter plots of latent versus physical distance for each axis. Each point in the plot corresponds to a sampled point along an axis trajectory. Minor variance in latent distance at identical physical positions (std $\approx$ 0.01) reflects video rendering artifacts---including compression noise and lighting flicker---rather than meaningful representational uncertainty. Along all three axes, a clear non-linear relation between the physical and latent distance is seen: smaller latent errors co-occur with small Cartesian position errors. The Pearson correlation is plotted for reference. Interestingly, points that are practically identical in terms of position error have significant non-zero latent representation distance, not much lower than 0.3 for any of the sampled points. This observation implies that the encoder is sensitive to even very small differences in the visual appearance of the encoded frames.

Notably, the fitted trend lines reveal systematic differences in slope across axes. With latent distance on the horizontal axis and physical distance on the vertical axis, the slope indicates how much physical progress toward the goal is achieved per unit reduction in latent distance. The $y$-axis exhibits the steepest slope, while the $x$- and $z$-axes have shallower slopes. This ordering---$y > z > x$ in slope---directly mirrors the observed planning performance ($y > z > x$). The axis along which latent progress most efficiently translates to physical progress is also the axis for which CEM planning succeeds best. This suggests that the encoder's representational structure along a given direction may influence the effectiveness of CEM optimization: when a unit decrease in latent cost yields larger physical displacement, sampling and elite selection more readily identify actions that make meaningful progress toward the goal.

\begin{figure}[H]
    \centering
    \includegraphics[width=\linewidth]{Pictures/Experiments/01_zero_shot_single_axis_reaching/latent_physical_correlation_by_axis.png}
    \caption{Zero-shot correlation between latent $\ell_1$ distance and physical Euclidean distance to goal in the pretrained V-JEPA 2 encoder's representation space for single-axis reaching trajectories. Each subplot shows scatter plots of all frame-to-goal pairs from five repeated episodes (distinguished by color) moving along the $x$-, $y$-, and $z$-axes respectively. Linear fits reveal differing slopes across axes: the $y$-axis exhibits the steepest slope, indicating that latent progress translates most efficiently to physical progress along this axis. The pretrained encoder exhibits strong positive correlations (Pearson $r = 0.90$--$0.92$) between latent and physical distance across all axes.}

    \label{fig:latent_physical_alignment}
\end{figure}


% See result plots and data in/home/s185927/thesis/experiments/01_zero_shot_single_axis_reaching/follow_up_experiments/latent_physical_distance_alignment/latent_physical_correlation_by_axis.png
% /home/s185927/thesis/experiments/01_zero_shot_single_axis_reaching/follow_up_experiments/latent_physical_distance_alignment/latent_physical_correlation_data.npz




While informative of the relation between the physical and latent space objectives, Figure \ref{fig:latent_physical_alignment} does not provide much insight into the drivers of the poor zero-shot planning results. To provide a more detailed view, Figure \ref{fig:energy_trajectory} shows slices of the energy landscapes for a single planning episode along the $x$-axis, similar to the ones illustrated in Figure \ref{fig:exp01_zero_shot}, alongside both the captured frames and the stepwise position error. The energy landscape visualizes how the action-conditioned predictor $P_\phi$ evaluates candidate actions at each planning step. For a given state $(s_k, z_k)$, each heatmap displays the predicted latent distance to the goal for every possible action $(\Delta x, \Delta y)$ within the $\pm 7.5$,cm action bounds:
\begin{equation}
\mathcal{E}(\hat{a}_k;z_k, s_k, z_g) = \frac{1}{TD}\| P_{\phi}(\hat{a}_k; s_k, z_k) - z_g \|1,
\end{equation}
where $P\phi(\hat{a}_k; s_k, z_k)$ is the predictor's imagined future latent state after taking action $\hat{a}_k$, and $z_g$ is the latent representation of the goal image. Darker regions indicate actions that the model predicts will bring the latent state closer to the goal, and the CEM optimizer concentrates its sampling toward these low-energy regions over successive iterations.

Overlaid on each landscape are two arrows: the red arrow indicates the ground-truth optimal direction, computed as the straight-line vector from the current end-effector position to the goal position, while the green arrow shows the action ultimately selected by CEM. The angular offset between these arrows reveals a bias in the predictor's learned dynamics. Although the task requires predominantly positive $x$-axis motion, the energy minima in the early planning steps favor actions with substantial positive $y$-components, causing unintended forward motion. In later steps, the planner partially compensates by selecting actions with negative $y$-components, but this corrective behavior is insufficient to fully recover from the initial deviation. This confirms the observation from Figure~\ref{fig:exp01_zero_shot} that planning along the $x$-axis leads to trajectories with substantial unintended $y$-axis displacement. The planner is functioning correctly, it successfully identifies and selects actions in low-energy regions, but the predictor's internal model of action consequences is misaligned with the true environmental dynamics along this axis. This bias likely reflects the distribution of actions in the DROID pretraining data, where pure lateral motions may be underrepresented relative to combined forward-lateral reaching movements.


\begin{figure}[H]
    \centering
    \includegraphics[width=\linewidth]{Pictures/Experiments/01_zero_shot_single_axis_reaching/energy_trajectory_ep0.png}
\caption{Energy landscape trajectory for a single episode of zero-shot single-axis reaching along the $x$-axis. Top row: Observed frames $x_k$ at each step $k \in \{0, \ldots, 5\}$, with the goal frame $x_g$ shown in the final, middle row. Middle row: Energy landscapes showing the predicted latent distance $\frac{1}{TD}\| P_\phi(\hat{a}_k; s_k, z_k) - z_g \|_1$ as a function of candidate actions $\hat{a}_k = (\Delta x, \Delta y, 0)$, here only considering the $x$- and $y$-components. The red arrow indicates the direction that would move the end-effector closest to the goal position, the green arrow shows the energy-minimizing action updates along the $x$-, and $y$-axis as chosen by CEM planning, and the white circle marks the current position (zero displacement). Bottom row: Euclidean position error $\|p_k - p_g\|_2$ between the end-effector and goal position at each step, with the 5\,cm success threshold indicated by the dashed red line. A clear misalignment between chosen and optimal directions is visible in the early planning steps, where the planner selects actions with substantial positive $y$-components despite the task requiring primarily $x$-axis motion. In later steps, the arrows gradually align and action magnitudes decrease as the energy minima shift closer to the origin, yet this improved alignment yields only modest reduction in Cartesian position error.}
    \label{fig:energy_trajectory}
\end{figure}

% See above plot in /home/s185927/thesis/experiments/01_zero_shot_single_axis_reaching/left_cam_cem_optimized/reach_along_x/energy_trajectory_ep0.png