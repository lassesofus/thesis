\section{Zero-shot reaching tasks}\label{sec:zero-shot_reaching}
The first experiment evaluates the zero-shot performance of the pretrained V-JEPA-2-AC model on simple reaching tasks in the simulated RoboHive environment described in Section~\ref{sec:robohive_sim}. Performance on these tasks provides a baseline assessment of the model’s ability to plan from purely visual goal images in an unseen domain. Establishing this baseline is essential for later isolating failure modes related to language grounding, rather than conflating them with errors arising from the sim-to-real gap.


\subsection{Methods}

\subsubsection{Task setup}
The zero-shot evaluation consists of single-axis reaching tasks along the $x$-, $y$-, and $z$-axes. For a given axis, one episode of a reaching task proceeds through three phases:

\begin{enumerate}
\item \textbf{Initialization.}
The robot arm is initialized at a starting configuration
$s_0 = (p_0, r_0, g_0)^{\top}$, where $p_0$ denotes the end-effector position,
$r_0$ its orientation, and $g_0$ the gripper state.

\item \textbf{Goal construction.}
Using the inverse kinematics (IK) solver (section \ref{sec:IK}), the arm is moved to a goal configuration $s_g$
defined by a 20\,cm Cartesian offset from $p_0$ along the selected axis.
A goal image $x_g$ is captured at this configuration, after which the arm is returned
to the starting pose $s_0$ using IK control to then capture the starting frame, $x_0$.

\item \textbf{Action planning.}
Starting from $s_0$, action planning is performed using a receding-horizon scheme, initially over five planning steps.
At each step $k$, the CEM optimizer selects an action $a_k^{\star}$ as described in
Section~\ref{sec:CEM}, which is executed using inverse kinematics to reach the next state $s_{k+1}$ and capture the next frame $x_{k+1}$. CEM planning is performed using a fixed set of hyperparameters, summarized in Table~\ref{tab:cem_config}. Notably, CEM optimizes a single-step action sequence (planning horizon $H = 1$), and replanning is performed after executing each action. 

In this simplified setting, the action space is limited to consist solely of Cartesian end-effector position updates. End-effector orientation $r_k$ and gripper state $g_k$ are therefore held fixed throughout planning and execution.

\begin{table}[ht]
\centering
\caption{CEM optimizer configuration for zero-shot reaching tasks.}
\label{tab:cem_config}
\begin{tabular}{lrl}
\toprule
\textbf{Parameter} & \textbf{Value} & \textbf{Description} \\
\midrule
Planning horizon ($H$) & 1 & Number of forward prediction steps \\
Trajectories ($J$) & 800 & Number of candidate trajectories per iteration \\
Top-$k$ ($K$) & 10 & Size of elite set \\
CEM iterations ($I$) & 10 & Number of refinement iterations \\
Momentum (mean) & 0.15 & Smoothing for mean update \\
Momentum (mean, gripper) & 0.15 & Smoothing for gripper mean update \\
Momentum (std) & 0.75 & Smoothing for std update \\
Momentum (std, gripper) & 0.15 & Smoothing for gripper std update \\
Max norm & 0.075 & Maximum action magnitude (m) \\
\bottomrule
\end{tabular}
\end{table}


\subsubsection{Evaluation}
For each planning step, both the Cartesian position error $\lVert p_k - p_g \rVert_2$ and the mean latent-space distance $\frac{1}{TD}\lVert z_k - z_g \rVert_1$ between encoded observations ($z_k = E_{\theta}(x_k) \in \mathbb{R}^{T\times D}$) are recorded, where $T = 256$ is the number of spatial tokens and $D = 1408$ is the embedding dimension of the frozen encoder. Each axis-specific experiment is repeated for $N=10$ episodes to account for stochasticity induced by CEM.

An episode is considered successful if the final Cartesian position error after five planning steps is below 5\,cm. This threshold corresponds to the practical precision floor of the IK and actuator execution pipeline (Appendix~\ref{app:ik_error}). No camera-position calibration is initially applied, in order to evaluate the raw zero-shot transfer behavior of the model under a fixed monocular viewpoint.