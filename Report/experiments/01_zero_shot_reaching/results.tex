\subsection{Results}
The results of the zero-shot reaching experiments are shown in Figure~\ref{fig:exp01_zero_shot}. For reaching along the $y$-axis, the planner achieves strong performance, reducing Cartesian position error from 20\,cm to approximately 2\,cm on average after five planning steps. Error reduction is largely monotonic with very low variance across episodes (final errors ranging from 1.8\,cm to 2.6\,cm), indicating stable and repeatable planning behavior under stochastic CEM optimization. For the $z$-axis, performance is similarly strong: the average error decreases from 20\,cm to approximately 3\,cm, with moderate inter-episode variance (final errors ranging from 2.5\,cm to 4.4\,cm). For the $x$-axis, performance is weakest: the average error decreases from 20\,cm to approximately 7\,cm (final errors ranging from 6.0\,cm to 7.9\,cm).

The average latent $\ell_1$ distances, shown in the bottom row of Figure~\ref{fig:exp01_zero_shot}, reveal that the efficiency of converting latent-space progress into Cartesian-space progress varies across axes. The $z$-axis exhibits the largest latent distance reduction, from approximately 0.49 to 0.23 over five planning steps (roughly 53\%), accompanied by strong Cartesian improvement. The $y$-axis achieves the strongest Cartesian error reduction with a moderate net decrease in latent $\ell_1$ distance (from approximately 0.54 to 0.43, roughly 22\%), with the average distance initially \emph{increasing} during the first two planning steps before decreasing. The $x$-axis undergoes a substantial latent distance reduction (from approximately 0.57 to 0.41, roughly 28\%) yet achieves only modest Cartesian improvement. The $y$-axis demonstrates the highest conversion efficiency, achieving 90\% Cartesian error reduction with only 22\% latent reduction, while the $x$-axis shows the lowest efficiency despite comparable latent progress.

\begin{figure}[H]
    \centering
    \includegraphics[width=\linewidth]{Pictures/Experiments/01_zero_shot_single_axis_reaching/consolidated_analysis.png}
    \caption{Results of zero-shot single-axis reaching tasks with V-JEPA 2-AC. Each column shows reaching along a different axis ($x$, $y$, $z$): start ($x_0$) and goal ($x_g$) frames (row 1), captured frames from the executed trajectory of the first episode with per-step action deltas $a^i_{t-1}$ in cm (row 2), step-wise Cartesian position error (row 3), and latent $\ell_1$ distance showing both measured values and predicted values from the world model (row 4). Error bars: $\pm1$ std ($N=10$ episodes).}
    \label{fig:exp01_zero_shot}
\end{figure}

