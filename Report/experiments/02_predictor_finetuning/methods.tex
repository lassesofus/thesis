\section{Predictor fine-tuning on simulated x-axis trajectories}\label{sec:x_axis_fine_tuning}

To investigate whether the poor zero-shot planning performance of V-JEPA~2-AC in the simulated RoboHive environment can be mitigated through domain adaptation, the action-conditioned predictor is fine-tuned on trajectories collected directly in simulation. The experiment focuses exclusively on motion along the $x$-axis, where zero-shot planning exhibited the largest degradation. Only the predictor is updated, while the visual encoder remains frozen, following the adaptation strategy proposed in the original V-JEPA~2-AC work.

\subsection{Methods}

\subsubsection{Data generation}
The robot arm is initialized at a fixed starting configuration $s_0 = (p_0, r_0, g_0)^{\top}$. A dataset of 1000 single-axis reaching trajectories is generated by sampling a target displacement along the $x$-axis:
\begin{equation*}
    \Delta p^x \sim U(0.05\,\text{m}, 0.3\,\text{m}),
\end{equation*}
with corresponding target position
\begin{equation*}
    p_g^x = p_0^x + \Delta p^x.
\end{equation*}

Each trajectory is executed over a 4.5\,s horizon using inverse kinematics with min-jerk smoothing. RGB video is recorded at 30\,fps together with robot state information (end-effector pose and gripper state). During training, videos are temporally downsampled to 4\,fps to match the V-JEPA~2 training configuration.

\subsubsection{Data splits}
The dataset is split into 640 training, 160 validation, and 200 held-out test trajectories. To assess data efficiency, the predictor is trained on 25\%, 50\%, 75\%, and 100\% of the 640 training samples (160, 320, 480, and 640 trajectories respectively).

\subsubsection{Training procedure}
Only the action-conditioned predictor is fine-tuned, while the encoder is kept frozen. The predictor is initialized from the official V-JEPA~2-AC checkpoint. Training uses AdamW with a learning rate of $10^{-4}$ (approximately 4$\times$ lower than pretraining), weight decay of 0.04, and a cosine annealing schedule with 5 epochs of linear warmup.

Early stopping is applied with a patience of 10 epochs and a minimum validation improvement threshold of 0.0001. Due to the small size of the simulation dataset and to avoid overfitting to transformations not present at inference time, all data augmentation is disabled: no random resizing, no aspect-ratio jitter, no horizontal flips, and no motion shift.

\subsubsection{Evaluation protocol}
Evaluation follows the same three-phase procedure as the zero-shot experiments: (1) move to the target configuration and capture a goal image, (2) return to the starting configuration, and (3) perform 5 steps of CEM-based action planning. Performance is measured as the Euclidean distance between the final end-effector position and the target. An episode is considered successful if the final distance is below 5\,cm. Each model is evaluated on the 200 test trajectories and compared against the zero-shot Meta V-JEPA~2-AC baseline. 